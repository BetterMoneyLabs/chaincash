\documentclass{article}   % list options between brackets

\usepackage{color}
\usepackage{graphicx}
%% The amssymb package provides various useful mathematical symbols
\usepackage{amssymb}
%% The amsthm package provides extended theorem environments
%\usepackage{amsthm}
\usepackage{amsmath}

\usepackage{listings}

\usepackage{hyperref}

\usepackage{systeme}

\def\shownotes{1}
\def\notesinmargins{0}

\ifnum\shownotes=1
\ifnum\notesinmargins=1
\newcommand{\authnote}[2]{\marginpar{\parbox{\marginparwidth}{\tiny %
  \textsf{#1 {\textcolor{blue}{notes: #2}}}}}%
  \textcolor{blue}{\textbf{\dag}}}
\else
\newcommand{\authnote}[2]{
  \textsf{#1 \textcolor{blue}{: #2}}}
\fi
\else
\newcommand{\authnote}[2]{}
\fi

\newcommand{\knote}[1]{{\authnote{\textcolor{green}{kushti notes}}{#1}}}
\newcommand{\mnote}[1]{{\authnote{\textcolor{red}{scalahub notes}}{#1}}}

\usepackage[dvipsnames]{xcolor}
\usepackage[colorinlistoftodos,prependcaption,textsize=tiny]{todonotes}


% type user-defined commands here
\usepackage[T1]{fontenc}

\usepackage{flushend}


\newcommand{\cc}{ChainCash}

\newcommand{\ma}{\mathcal{A}}
\newcommand{\mb}{\mathcal{B}}
\newcommand{\he}{\hat{e}}
\newcommand{\sr}{\stackrel}
\newcommand{\ra}{\rightarrow}
\newcommand{\la}{\leftarrow}
\newcommand{\state}{state}

\newcommand{\ignore}[1]{} 
\newcommand{\full}[1]{}
\newcommand{\notfull}[1]{#1}
\newcommand{\rand}{\stackrel{R}{\leftarrow}}
\newcommand{\mypar}[1]{\smallskip\noindent\textbf{#1.}}

\begin{document}

\title{ChainCash - elastic peer-to-peer money creation via trust and blockchain assets}
\author{kushti, scalahub}


\maketitle

\begin{abstract}
In this paper we introduce ChainCash, a protocol to create money in self-sovereign way via trust or collateral. The
protocol allows for elastic money creation in peer-to-peer environment, without any authority, being it authoritarian,
democractic or algorithmic. For that, acceptance of notes created by other peers is an individual choice.
\end{abstract}


\section{Introduction}

Currently, most of monetary value is created by private banks~(often, offshore banks as in so-called "eurodollar" system~\cite{machlup1970euro}) following central banks requirements. As an alternative, starting with Bitcoin~\cite{nakamoto2008peer} launch in 2009, a lot of cryptocurrencies 
and DeFi applications on top of public blockchains are experimenting with algorithmic money issuance. As another option, we also have alternative, usually local, monetary systems, such as LETS~(local exchange trading systems~\cite{williams1996new}), timebanks, local government currencies~\cite{unterguggenbercer1934end}, and so on. Control in traditional fiat monetary systems is possessed by big players~(with rich getting richer effect) creating money in non-transparent ways~(especially in offshore parts) without reasonable limits, on the other side, they have better supply elasticity. Cryptocurrencies~(and, sometimes, application and other tokens on top of public blockchains) are usually more decentralized, have known emission schedule, which makes them appealing assets, on the other hand, supply is disconnected from economic activity and not elastic. Local currencies usually considered more fair in segniorage distribution than fiat currencies, they are successfully boosting local economies often, but, in opposite to fiat and crypto-currencies, they are not global.

In this paper, we propose \cc{}, a new global kind of money, with decentralized issuance, elastic supply, and also collective backing~(not necessarily though) by collateral.


We consider money here via its medium-of-exchange property. For existing currencies, there are usually many options to represent value, such as coins, paper or plastic banknotes, digital records in different ledgers, etc. For \cc{}, we define money as a set of digital notes, each has some nominal~(not fixed but arbitrary on our case). Value of a note is nominated in some existing widely recognizeable unit-of-account, such as gold. 

We consider that an economy is consisting of known agents $a_1, ..., a_n$. Then we can define medium-of-exchange property of money via a set of agents accepting monetary objects~(i.e. notes). Usually, set of agents accepting money~(local or foreign currency) is fixed. That is, for every monetary object~(e.g. a note), an agent is accepting it as a mean of incoming payment, or reject. In opposite, for \cc{} money, similarly to~\cite{saito2003peer}, the set is individual for a note, so when agent $a_i$ sees a note $n$, it applies its personal predicate $P_i(n)$ to decide whether to accept the note or decline it. 

Then how notes are different in case of \cc{}? We consider that every note is collectively backed by all the previous spenders of the note. Every agent may create reserves to be used as collateral. When an agent spends note, whether received previously from another agent or just created by the agent itself, it is attaching its signature to it. A note could be redemeed at any time against any of reserves of agents previously signed the note. Redemption fee should be paid though, and the fee is incentivizing reserves provision and also using the notes instead of redeeming them. We allow an agent to issue and spend notes without a reserve. It is up to agent's counter-parties then whether to accept and so back an issued note with collateral or agent's trust or not.

As an example, consider a small gold mining cooperative in Ghana issuing a note backed by (tokenized) gold. The note is then accepted by the national government as mean of tax payment. Then the government is using the note~(which is now backed by gold and also trust in Ghana government, so, e.g. convertible to Ghanaian Cedi as well) to buy oil from a Saudi oil company. Then the oil company, having its own oil reserve also, is using the note to buy equipment from China. Now a Chinese company has a note which is backed by gold, oil, and Cedis. 

Agent's note quality estimation predicate $P_i(n)$ is considering collaterals and trust of previous spenders. Different agents may have different 
collateralization estimation algorithm~(by analyzing history of the single note $n$, or all the notes known), different whitelists, blacklists, or trust scores assigned to previous spenders of the note $n$ etc. So in general case payment sender first need to consult with the receiver on whether the payment~(consisting of one or multiple notes) can be accepted. However, in the real world likely there will be standard predicates, thus payment receiver~(e.g. an online shop) may publish its predicate (or just predicate id) online, and then the payment can be done without prior interaction.

We propose to implement \cc{} monetary system on top of a blockchain as:

\begin{itemize}
  \item{} Blockchain provides an instant solution for public-key infrastructure.
  \item{} Public blockchain allows for a global ledger solution with minimal trust assumptions~\cite{kya}.
  \item{} As a consequence, global public ledger allows for simple analysis of the notes in existence.
  \item{} Smart contracts minimize trust issues in payment execution and redemption. If native blockchain currency and assets on top of it~(such as algorithmic stablecoins) used in reserves, trust issues in redemption could be eliminated at all. If tokenized real-world commodities and fiat currencies~(e.g. USDT) are used in reserves, redemption could not be completely trustless~(as smart contracts do not have power off the chain), but at least there is transparent accounting in on-chain part of redemption. 
\end{itemize}

We use Ergo as a Proof-of-Work blockchain to implement \cc{}, as it is built on minimal trust assumptions~\cite{kya}, and UTXO transactional model as well as AVL+ trees support are making notes implementation feasible.


\section{\cc{} Implementation}

To implement \cc{} on top of a UTXO cryptocurrency, such as Ergo, two contracts are needed, a note contract and a reserve contract. Reserve contract locks ERG native tokens on top of Ergo blockchain and allow to redeem native or custom tokens when a note is presented. Note contract ensures that the note has proper history, that is, on every spending a valid signature of corresponding reserve is added. It also and allows for a note to be split into two parts~(payment and change), and allows for note redemption.

\subsection{Layer1 implementation}
\label{sec:l1-impl}

On-chain note and reserve contracts are available in~\cite{contracts}.


\subsection{Layer2 implementation}

\cc{} allows for efficient Layer-2 implementation \knote{finish}


\subsection{Offchain and on-chain part}

Off-chain part is the most important piece of \cc{} system, as it allows for recognizing possibly different forms of note and reserve contracts, and also do note value estimation. 


\section{Applications}

\cc{} could be seen as a powerful foundation for other monetary systems, and we are going to show it in this section. 

\subsection{LETS}

To implement a local exchange trading system on top of \cc{}, every LETS member needs to whitelist every over member, so they will accept notes of each other regardless reserves backing the notes, and thus LETS can create money within the community (the LETS circle) with no limits. On the other hand, unlike traditional LETS, notes can circulate outside the LETS circle easily as well. Implementations may vary from LETS members whitelisting unconditionally only notes issued by other members to members whitelisting notes ever signed by LETS members. 


\subsection{Local Currencies}

A local or even national government may issue notes and enforce their acceptance within its jurisdiction by enforcing economic agents to accept notes issued or spend by the government. As well as in a LETS implementation, enforced acceptance rules may vary. 

Often local currencies are introducing redemption fee, to promote local usage. In \cc{}, similar goals can be achieved via modifying the reserve contract in a way that non-locals need to pay redemption fee while locals need not, alternatively, the note contract could be modified in a way that spending to non-local addresses incurs a fee. Local currencies are often associated with demurrage, after well-known Woergl experiment~\cite{unterguggenbercer1934end}. Demmurage could be implemented by modifying note contract. However, modifying note contract makes notes locked by it less appealing to others, but that is common for local currencies already.

\subsection{Multilateral Trade-Credit Set-off}

Multilateral Trade-Credit Set-off~\ref{mtcs} is a technique which allows invoices in closed loops to be cleared against one another.
In \cc{}, it is possible to clear mutual debts by just burning atomically tokens backed by counter-partis in a single
transaction. This will allow them to issue more credit.

\section{\cc{} Advantages}

\section{\cc{} Drawbacks}

ChainCash notes are non-fungible, they share the same unit-of-account, each note has unique backing. This prevents ChainCash usage
in many DeFi applications, such as liquidity pools, lending pools etc.


\section{Possible Extensions And Clients Flexibility}

Basic Layer-1 implementation described in Section~\ref{sec:l1-impl} is good for starters, but can be extended in many ways,
and we list some in this section.

First of all, we note that it is possible to add new features without need for the whole network to update. New features,
such as new reserve and note contracts, can be proposed in form of CCIPs~(ChainCash Improvement Proposals). Then ChainCash
client may support new features, means new forms of money defined by them. If client is asked to accept a note with unknown
contract, or a note backed by unknown contract, it will refuse to accept the note just.


\newpage
\bibliography{sources}
\bibliographystyle{ieeetr} 

\end{document}
